\documentclass{article}[12pt]
\usepackage{fullpage}
% Packages
\usepackage{amsfonts}
\usepackage{amsmath}
\usepackage{amsthm}

\theoremstyle{plain}
\newtheorem{lemma}{Lemma}
\newtheorem{claim}[lemma]{Claim}
\newtheorem{theorem}[lemma]{Theorem}
\newtheorem{corollary}[lemma]{Corollary}

\newcommand{\vx}{\vec{x}}
\newcommand{\vr}{\vec{r}}
\newcommand{\vp}{\vec{p}}
\newcommand{\vW}{\vec{W}}
\newcommand{\vl}{\vec{\ell}}
\newcommand{\grad}{\bigtriangledown}

\title{Lagrangian analysis for two points}
\author{Yoav Freund}
\begin{document}
\maketitle

\newcommand{\paren}[1]{\left( #1 \right)}
\newcommand{\pot}[2]{\phi\left( #1,#2 \right)}
\newcommand{\potExpr}[2]{\exp\left( \frac{#1^2}{3 #2}\right)}
\newcommand{\dT}[1]{D(#1)}
\newcommand{\dR}[1]{R(#1)}
\newcommand{\dA}[1]{A(#1)}
\newcommand{\dB}[1]{B(#1)}
\newcommand{\thirdCol}[2]{#2 \left(1+ \frac{2}{3} #1^2\right) \potExpr{#1}{}}
\newcommand{\thirdColSimp}[2]{#2 \paren{\frac{1}{#1}+ \frac{2}{3} #1}\potExpr{#1}{}}

We define the scalar potential of a single expert whose cumulative
regret at time $t$ is $x$ to be:
\[
\pot{x}{t} \doteq 
\begin{cases} 
e^{\frac{x^2}{3t}} &\mbox{if } x \geq 0 \\
1 & \mbox{otherwise}
\end{cases} 
\]

In what follows we will always assume that $x\geq 0$ so that only the
first case matters.
\[
\pot{x}{t} = \potExpr{x}{t}
\]

We restrict our attention to two experts, defined by the five parameters
$(x_1,x_2,p_1,p_2,q)$. Where $x_1,x_2 \geq 0$ are the cumulative
regrets. $0 \leq p_1,p_2 \leq 1$ are the probabilities associated with
a loss of $1$, and $(q,1-q)$ are the probabilities associated with the
first and the second experts respectively.

We assume that $t$ has been folded into $x_1,x_2$ so that the initial
time is $t=1$. The final time is $t=1+d^2$ for some small $d>0$.
\newcommand{\changeExpr}[5]{#2 \potExpr{\paren{#1 #4 #3}}{\paren{1+#3^2}}+(1-#2) \potExpr{\paren{#1 #5 #3}}{\paren{1+#3^2}}-\potExpr{#1}{} }

We denote the increase in potential for a particular $x$ by 
\begin{eqnarray*}
\psi(x,p,d,L)&=&
p \pot{x+(1-L)d}{1+d^2}
+(1-p) \pot{x+(-1-L)d}{1+d^2}
-\pot{x}{1} \\
&=&
\changeExpr{x}{p}{d}{+(1-L)}{+(-1-L)}
\end{eqnarray*}

We define a single expert $x_3,p_3$ that can replace the two experts
while conserving two quantities: the average potential and the average
loss. Conservation of average potential gives:
\begin{equation}
q\potExpr{x_1}{}+(1-q)\potExpr{x_2}{} = \potExpr{x_3}{} \label{eqn:C1}
\end{equation}
from which we can derive the value of $x_3$. Conservation of average
loss gives:
\begin{equation}
q p_1 x_1 \potExpr{x_1}{}+(1-q) p_2 x_2 \potExpr{x_2}{} = p_3 x_3 \potExpr{x_3}{}
\label{eqn:C2}
\end{equation}
from which we can derive $p_3$. The conservation of average loss means
also that $L=p_3\cdot 1 + (1-p_3) \cdot (1-p_3) = 2p_3-1$.

Our goal is to show that the potential never increases. The step we
take towards this goal is to show that combining $(x_1,p_1)$ and
$(x_2,p_2)$ into $(x_3,p_3)$ will always increase the future
potential.

In symbols, that is:
\begin{eqnarray}
T(s) &= q& \paren{\changeExpr{x_1}{p_1}{d}{-2p_3}{+(2-2p_3)}} \\
&+
(1-q)&\paren{\changeExpr{x_2}{p_2}{d}{-2p_3}{+(2-2p_3)}} \notag \\
&-
&\paren{\changeExpr{x_3}{p_3}{d}{-2p_3}{+(2-2p_3)}} \notag
\end{eqnarray}

The function $T(s)$ is defined over the set $A\subset R^8$.
A generic element of $A$ is denoted by
$s=\langle x_1,x_2,x_3,p_1,p_2,p_3,q,d \rangle \in A$. The conditions
that define $A$ are:
\begin{enumerate}
\item $d>0$ is the ``step'' variable.
\item $1 \geq  p_1,p_2,p_3,q \geq 0$ are mixture proportions.
\item $x_1,x_2,x_3 \geq 2p_3 d$ are the ``positions'' of the three
  experts. 
\item The constraints given in Equations~(\ref{eqn:C1})
  and~(\ref{eqn:C2}) hold. 
\end{enumerate}

We want to show that:
\begin{claim}
\[
\sup \left\{T(s)\left| s \in A \right.\right\} \leq 0
\]
\end{claim}

\section{Restricting to one degree of freedom}
Suppose that out of the six degrees of freedom 
we fix five: $x_3,p_1,p_2,q,d$, we and vary one: $x_1$. 
Equation~(\ref{eqn:C1}) gives us $x_2$ (or that there is no solution)
and Equation~(\ref{eqn:C2}) gives us $p_3$. 

The claim is that the maximum of $T$ as we vary $x_1$,(which
determines $x_2,p_3$) is always achieved when $x_1=x_2=x_3$.

\section{Using the Lagrangian}
One approach that seems promising is to use the Lagrangian. We
redifine Equations~(\ref{eqn:C1}) and~(\ref{eqn:C2}) using variables $C_1,C_2$:
\begin{equation}
C_1\doteq q \potExpr{x_1}{}+(1-q)\potExpr{x_2}{} -\potExpr{x_3}{},\;\;\; C_1=0
\end{equation}
\begin{equation}
C_2\doteq p_1 x_1 \potExpr{x_1}{}+(1-q) p_2 x_2 \potExpr{x_2}{} - p_3
x_3 \potExpr{x_3}{},\;\;\;C_2=0
%\label{eqn:C2}
\end{equation}
We then define the lagrangian:
\newcommand{\Lag}{{\cal L}}
\[
\Lag \doteq T+\lambda_1 C_1 +\lambda_2 C_2
\]

The condition we are seeking is that
\newcommand{\parti}[1]{\frac{\partial}{\partial #1}} 
\[
\parti{x_1} \Lag=0,\;\;
\parti{x_2} \Lag=0,\;\;
\parti{x_3} \Lag=0,\;\;
\]
Where $\lambda_1\neq 0$ or $\lambda_2 \neq 0$.

Another, more revealing way to state these conditions is to say that
the rank of the following matrix is two:
\newcommand{\Mlag}{{\cal M}}
\begin{equation}
\Mlag \doteq
\begin{pmatrix}
 \parti{x_1} T & \parti{x_1} C_1 & \parti{x_1} C_2\\
 \parti{x_2} T & \parti{x_2} C_1 & \parti{x_2} C_2\\
 \parti{x_3} T & \parti{x_3} C_1 & \parti{x_3} C_2
 \end{pmatrix} 
\end{equation}

With the help of Maple we find that 
\begin{equation}
\Mlag=
\paren{
\begin{array}{c|c|c}
q \dT{x_1,p_1} & q x_1 \potExpr{x_1}{} &
q \thirdCol{x_1}{p_1} \\
\hline
(1-q) \dT{x_2,p_2} & (1-q) x_2 \potExpr{x_2}{} &
(1-q) \thirdCol{x_2}{p_2} \\
\hline
- \dT{x_3,p_3} & - x_3 \potExpr{x_3}{} &
- \thirdCol{x_3}{p_3} \\
\end{array}
}
\label{eqn:Matrix}
\end{equation}
Where
\begin{equation}
\dT{x,p} \doteq 
\frac{p(x+(2-2p_3)d)}{1+d^2}\exp\paren{\frac{\paren{x+(2-2p_3)d}^2}{3(1+d^2)}}
+
\frac{(1-p)(x-2p_3d)}{1+d^2}\exp\paren{\frac{\paren{x-2p_3d}^2}{3(1+d^2)}}
- x e^{x^2 / 3}
\label{eqn:DT}
\end{equation}

Multiplying a row or a column by a non-zero constant does not change
the rank. Therefor we can devide the first row by $q$, the second row
by $(1-q)$ and reverse the sign in the last row to get:
\[
\Mlag' =
\paren{
\begin{array}{c|c|c}
\dT{x_1,p_1} & x_1 \potExpr{x_1}{} &
\thirdCol{x_1}{p_1} \\
\hline
\dT{x_2,p_2} & x_2 \potExpr{x_2}{} &
\thirdCol{x_2}{p_2} \\
\hline
\dT{x_3,p_3} & x_3 \potExpr{x_3}{} &
\thirdCol{x_3}{p_3}
\end{array}
}
\]

Suppose $x_1=x_2$. From condition~(\ref{eqn:C1}) we find that
$x=x_3=x_1=x_2$. and from condition~(\ref{eqn:C2}) we find that
\[
p_3 = q p_1 + (1-q) p_2
\]  
Plugging these into Equation~(\ref{eqn:Matrix})
\[
\Mlag'' =
\paren{
\begin{array}{c|c|c}
\dT{x,p_1} & x \potExpr{x}{} &
\thirdCol{x}{p_1} \\
\hline
\dT{x,p_2} & x \potExpr{x}{} &
\thirdCol{x}{p_2} \\
\hline
\dT{x,p_3} & x \potExpr{x}{} &
\thirdCol{x}{p_3} \\
\end{array}
}
\]

The sum of the first row times $q$ and the second row times $1-q$ is
equal to the third row. We therefor find that the rank of the matrix
is 2 and the point is indeed an extremal point.

What remains to be shown is that:
\begin{enumerate}
\item This extremal point is a maximum.
\item This extremal point is the global maximum (rather than one of
  the points on the boundry.
\item That the maximum under the condition $x_1=x_2=x_3$ is smaller
  than zero.~\footnote{I had a guess that the maximum is achieved when
    $p_1=p_2=p_3=1/2$, but kamalika showd that when $d>0$ the maximum
    is achieved elsewhere.}
\end{enumerate}


\section{$x_1 \neq x_2 \neq x_3$}
We simplify the matrix $\Mlag'$ further. We subtract Column 2 from
$\dT{x_i,p_i}$, and then divide row $i$ by $x_i$.
We get
\[\Mlag''' =
\paren{
\begin{array}{c|c|c}
\dR{x_1,p_1} & \potExpr{x_1}{} &
\thirdColSimp{x_1}{p_1} \\
\hline
\dR{x_2,p_2} & \potExpr{x_2}{} &
\thirdColSimp{x_2}{p_2} \\
\hline
\dR{x_3,p_3} & \potExpr{x_3}{} &
\thirdColSimp{x_3}{p_3}
\end{array}
}
\]
Where
\begin{eqnarray*}
\dR{x,p} &\doteq& 
\frac{p(x+(2-2p_3)d)}{x(1+d^2)}\exp\paren{\frac{\paren{x+(2-2p_3)d}^2}{3(1+d^2)}}
+
\frac{(1-p)(x-2p_3d)}{x(1+d^2)}\exp\paren{\frac{\paren{x-2p_3d}^2}{3(1+d^2)}}
\\
&=& \dA{x} + p \dB{x}
\end{eqnarray*}
Where
\[
\dA{x} \doteq 
\frac{x-2p_3d}{x(1+d^2)}\exp\paren{\frac{\paren{x-2p_3d}^2}{3(1+d^2)}}
\]
and
\[
\dB{x} \doteq 
\frac{x+(2-2p_3)d}{x(1+d^2)}\exp\paren{\frac{\paren{x+(2-2p_3)d}^2}{3(1+d^2)}}
- \frac{x-2p_3d}{x(1+d^2)}\exp\paren{\frac{\paren{x-2p_3d}^2}{3(1+d^2)}}
\]
\end{document}
